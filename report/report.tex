\documentclass[a4paper,oneside,11pt]{article}

\usepackage[english]{babel}
\usepackage[utf8]{inputenc}
\usepackage[T1]{fontenc}

% Gestion des marges
\usepackage{geometry}
\geometry{hmargin=2.5cm, top=2.5cm, bottom=2.5cm}
\usepackage{setspace}
\onehalfspacing

% Mathématiques
\usepackage{mathtools,amssymb,amsthm}

% Références
\usepackage{hyperref}

% Code
\usepackage{listings}
\lstset{breaklines=true}

% Images
\usepackage{graphicx}
\usepackage{caption}
\usepackage{subcaption}

% Gestion des en-têtes
\usepackage{lastpage}
\usepackage{hyperref}
\usepackage{fancyhdr}
\pagestyle{fancy}
\fancyhf{}
\setlength{\headheight}{15pt}
\fancyhead[L]{\bfseries ITR}
\fancyhead[R]{E~\textsc{de~Roffignac}, A~\textsc{Tauveron}}
\fancyfoot[C]{\thepage{}}
\renewcommand{\headrulewidth}{.4pt}
\renewcommand{\footrulewidth}{0pt}

\title{ITR\\Report}
\author{Edmond~\textsc{de Roffignac}, Aimery~\textsc{Tauveron-\,-Jalenques}}
\date{22 janvier 2019}

\begin{document}
\maketitle


\section*{TD1}
\paragraph{Question 1a} Done (\texttt{main\_td1a.cpp}).

\paragraph{Question 1b} Done (\texttt{main\_td1b.cpp}). The final value of the counter is the value passed as argument (and stored in variable \texttt{nLoops}).

\paragraph{Question 1c} Done (\texttt{main\_td1c.cpp}). When passing argument \texttt{pStop} to function \texttt{incr}, it must be declared as volatile.

\paragraph{Question 1d} A first solution for increasing the precision of $l$ would be to record more points, and then perform a linear regression on these data points (for example, using the least-squares methods\footnote{\href{https://en.wikipedia.org/wiki/Linear_least_squares}{en.wikipedia.org/wiki/Linear\_least\_squares}}).

\paragraph{Question 1e} The code is in files \texttt{timespec.h} and \texttt{timespec.cpp}. Some basic tests are implemented in \texttt{main\_td1e.cpp}.


\section*{TD2}
\paragraph{Question 2a} Done (\texttt{main\_td2a.cpp}). Executing the program with high values for \texttt{nLoops} and \texttt{nTasks}, we notice that the final value of the counter is less than the product $\texttt{nLoops}\cdot\texttt{nTasks}$. For small values, this is not the case. We conclude that there are concurrent accesses to the counter variable. This means that in some cases:
\begin{itemize}
  \item  one thread may reads the variable (its value is $i$),
  \item a second thread reads the variable (its value is still $i$),
  \item the first thread sets the variable to $i+1$,
  \item the second thread sets the variable to $i+1$ (again).
\end{itemize}

Note that an incrementation operation may never be atomic since it involves both a read and a write operation (sequentially).

\paragraph{Question 2c}

\paragraph{Question 2c} Done (\texttt{main\_td2c.cpp}). When not using a mutex, the inaccuracies observed in question 2a are still observed, which is consisted since we're essentially running the same program. When using a mutex, not such inaccuracies are observed and the final value of the counter is exactly $\texttt{nLoops}\cdot\texttt{nTasks}$. The version using a mutex is much slower than the one without (roughly by a factor of 20 to 30).

We conclude that a mutex allows us to protect the counter (or other ressources) from concurrent accesses, at the cost of decreased performance.

\section*{TD3}
\paragraph{Question 3a} The Chrono class can be found in files \texttt{Chrono.h} and \texttt{Chrono.cpp}. A sample main file, which illustrates some functionalities of the class, can be found in \texttt{main\_td3a.cpp}. As far as we can tell by counting seconds on a wrist-watch, it is working fine.

\paragraph{Question 3b}

\paragraph{Question 3c} The Calibrator class can be found in files \texttt{Calibrator.h} and \texttt{Calibrator.cpp}; the CpuLoop class in files \texttt{CpuLoop.h} and \texttt{CpuLoop.cpp}; the Looper class in files \texttt{Looper.h} and \texttt{Looper.cpp}. File \texttt{main\_td3c.cpp} makes the above classes.

\section*{TD4}
\paragraph{Question 4a} The custom thread class (derived from our \texttt{thread} class) can be found in files \texttt{IncrThread.h} and \texttt{IncrThread.cpp}. The main file is \texttt{main\_td4a.cpp}.

\paragraph{Question 4b} The custom thread class (derived from our \texttt{thread} class) can be found in files \texttt{SafeIncrThread.h} and \texttt{SafeIncrThread.cpp}. The main file is \texttt{main\_td4b.cpp}.

\paragraph{Question 4c} The Semaphore class can be found in files \texttt{Semaphore.h} and \texttt{Semaphore.cpp}. A producer-consumer pattern is implemented in \texttt{main\_td4c.cpp}. Our tests have shown that all produced tokens are consumed.

\paragraph{Question 4d} The Fifo template can be found in file \texttt{Fifo.h} (the use of a template is dictated by the necessity to be able to store any one type of data in the structure). A producer-consumer pattern is implemented in \texttt{main\_td4d.cpp}. Our tests have shown that all produced tokens are consumed.

\end{document}